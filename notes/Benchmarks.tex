\documentclass[12pt,letterpaper,notitlepage]{article}

\usepackage{epsfig,amsmath,amssymb,cite,color,multirow,ulem,bm}

\usepackage{graphicx,wasysym}
\usepackage{hyperref}
\usepackage{tabulary}

\usepackage{pdfpages}
\usepackage{xcolor}

\newcolumntype{K}[1]{>{\centering\arraybackslash}p{#1}}

\makeatletter
\newcommand{\thickhline}{%
    \noalign {\ifnum 0=`}\fi \hrule height 1pt
    \futurelet \reserved@a \@xhline
}
\newcolumntype{"}{@{\hskip\tabcolsep\vrule width 1pt\hskip\tabcolsep}}

\newcommand{\listintertext}{\@ifstar\listintertext@\listintertext@@}
\newcommand{\listintertext@}[1]{% \listintertext*{#1}
  \hspace*{-\@totalleftmargin}#1}
\newcommand{\listintertext@@}[1]{% \listintertext{#1}
  \hspace{-\leftmargin}#1}
\makeatother

\oddsidemargin 0cm
\evensidemargin 0cm
\marginparwidth 68pt
\marginparsep 10pt
\topmargin 0cm
\headheight 0pt
\headsep 0pt
\footskip 30pt
\textheight 22cm
\textwidth 16.5cm\textbf{}
\columnsep 10pt
\columnseprule 0pt

\newcommand{\tev}{\,\, \mathrm{TeV}}
\newcommand{\gev}{\,\, \mathrm{GeV}}
\newcommand{\mev}{\,\, \mathrm{MeV}}



\begin{document}

\begin{center}
\LARGE Scan over benchmarks
\end{center}

\vspace{1.0cm}
\begin{abstract}
\vspace{0.2cm}\noindent
The goal is to study more benchmarks and see, which directions in the parameter space would lead to the better signal-background discrimination.
\end{abstract}


The idea is to fix one of the parameters and to vary the rest in the following way (the graphical representation is swown on fig.\ref{fig:first-benchmarks}):
\begin{itemize}
  \item Keep $\Delta_m=m_C-m_l$ and $c \tau_C$ fixed (not changing the hardness of the leptons) and decrease $m_C$. This will increase the cross section and potentially $d_0$ (due to the slightly higher boost). Since we are in principle trying to explore the reach of our analysis to higher masses, we would suggest to first decrease the charged mass to 200 GeV, with a potential further decrease to 150 GeV.
  \item Keep $\Delta_m=m_C-m_l$ and $m_C$ fixed and increase  $c \tau_C$ (decrease the coupling r in code). This would show us, how much we win by making the displacement large. However, based on Table 4 from \cite{CMS:2016isf}, one can infer that the current $c \tau_C=2\ cm$ is probably the one with the highest sensitivity. In order to check this statement we propose to simulate events with $c \tau_C=10\ cm$ ($r=5e-9$ in code code).
  \item In order to check if we could reach for higher masses, we propose to have a look on the large mass splittings while fixing $c \tau_C$ and $m_C$. In principle, we could also slightly vary the PT-cut and look for similar effects. For now, we suggest to increase the mass splitting to $\Delta m=30$ GeV which would populate more the region above the current PT cut and hopefully make a better signal-background discrimination possible.
\end{itemize}

\begin{figure}[H]
\centering
\includegraphics[height=2.5in]{fig-benchmarks/ctau-m.png}
\includegraphics[height=2.5in]{fig-benchmarks/m-splitting.jpg}
\caption{\label{fig:first-benchmarks} Proposed directions in the parameter space. The current benchmark point is indicated by the cyan dot.}
\end{figure}

\begin{thebibliography}{6}

  \bibitem{CMS:2016isf}
    CMS Collaboration [CMS Collaboration],
    %``Search for displaced leptons in the e-mu channel,''
    CMS-PAS-EXO-16-022.


\end{thebibliography}

\end{document}
