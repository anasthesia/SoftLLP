\documentclass[12pt,letterpaper,notitlepage]{article}

\usepackage{epsfig,amsmath,amssymb,cite,color,multirow,ulem,bm}

\usepackage{graphicx,wasysym}
\usepackage{hyperref}
\usepackage{tabulary}

\usepackage{pdfpages}
\usepackage{xcolor}

\newcolumntype{K}[1]{>{\centering\arraybackslash}p{#1}}

\makeatletter
\newcommand{\thickhline}{%
    \noalign {\ifnum 0=`}\fi \hrule height 1pt
    \futurelet \reserved@a \@xhline
}
\newcolumntype{"}{@{\hskip\tabcolsep\vrule width 1pt\hskip\tabcolsep}}

\newcommand{\listintertext}{\@ifstar\listintertext@\listintertext@@}
\newcommand{\listintertext@}[1]{% \listintertext*{#1}
  \hspace*{-\@totalleftmargin}#1}
\newcommand{\listintertext@@}[1]{% \listintertext{#1}
  \hspace{-\leftmargin}#1}
\makeatother

\oddsidemargin 0cm
\evensidemargin 0cm
\marginparwidth 68pt
\marginparsep 10pt
\topmargin 0cm
\headheight 0pt
\headsep 0pt
\footskip 30pt
\textheight 22cm
\textwidth 16.5cm
\columnsep 10pt
\columnseprule 0pt

\newcommand{\tev}{\,\, \mathrm{TeV}}
\newcommand{\gev}{\,\, \mathrm{GeV}}
\newcommand{\mev}{\,\, \mathrm{MeV}}



\begin{document}

\begin{center}
\LARGE Singlet-triplet fermion dark matter\\
\Large LLP processes
\end{center}

\vspace{1.0cm}
\begin{abstract}
\vspace{0.2cm}\noindent
This file contains an estimate of $d_0$ dependence on the model parameters. In particular, I try to see how does the impact parameter change if the mediators $\chi^\pm$ are boosted
\end{abstract}

\section{Analytical estimations}


\begin{figure}[h!]
\centering
\includegraphics[width=0.6 \textwidth]{fig/d0}


\caption{\label{fig:d0} Definition of quantities used.}
\end{figure}

Let us look at the mediator $\chi^+$ decaying to the charged lepton $l^+$ and missing energy in transverse plane (x,y): fig.(\ref{fig:d0}). Let $\chi^+$ have a transverse length of flight $\vec{r}_T$ and the transverse momentum of the outgoing lepton is $\vec(p)_T$. The transverse impact parameter in then calculated as:
\begin{equation} \label{d0_init}
  d_0=r_T\ \sin (\alpha) = l_{\chi^+} \sin(\theta^+) \sin (\alpha)
\end{equation}
where $l_{\chi^+}$ is the full time of flight of the mediator, $\theta^+$ is the angle between mediator's trajectory and beam axis z, and $\alpha$ is defined as:
\begin{equation}
  \cos (\alpha) = \frac{\vec{p}_T \vec{r}_T}{p_T r_T}
\end{equation}

Let us now have a look on how these quantities change if the mediator mass changes. In this case a different amount of energy goes to the kinetic energy which means that $\chi^+$ would be boosted. Let us assume that, in addition,it doesn't change it's initial direction (this is probably a quite strong assumption due to PDF effects resulting in boosts in z direction only but for the simplicity I will stick to it here). Then $r_T$ is simply boosted by a gamma-factor: $r_T'=\gamma_T r_T$ and we only need to take care of the angle $\alpha$.

Let us project $\vec{p}_T$ to the $\vec{r}_T$ obtaining a parallel component $p_T^\parallel$ and a perpendicular one $p_T^\bot$. The latter doesn't change under the boost in $\vec{r}_T$ direction, but the parallel one is getting boosted:

\begin{align}
  &p_T^\bot '=p_T^\bot=p_T \sin(\alpha)\\
  &p_T^\parallel '=\gamma_T(\beta_T E+p_T^\parallel)=\gamma_T(\beta_T p+p_T \cos(\alpha))
\end{align}

where $E\simeq p$ is the energy of the lepton, all the prime components refer to the boosted system, and all the non-prime ones -- to the original system. The angle $\alpha$ then becomes ($p_T'=\sqrt{(p_T^\parallel')^2+(p_T^\bot')^2}=\sqrt{p_T^2 \sin^2(\alpha)+\gamma_T^2(\beta_T p+p_T \cos(\alpha))^2}$):

\begin{equation}
  \cos(\alpha')=\frac{\gamma_T r_T\ \gamma_T(\beta_T p+p_T^\parallel)}{\gamma_T r_T \sqrt{p_T^2 \sin^2(\alpha)+\gamma_T^2(\beta_T p+p_T \cos(\alpha))^2}}=\frac{\gamma_T(\beta_T p+p_T \cos(\alpha))}{\sqrt{p_T^2 \sin^2(\alpha)+\gamma_T^2(\beta_T p+p_T \cos(\alpha))^2}}
\end{equation}

Finally, we can use the relation $\beta_T=p_T/E\simeq p_T/p$ to simplify the above expression:
\begin{equation} \label{alpha_boosted}
  \cos(\alpha')=\frac{\gamma_T(1+\cos(\alpha))}{\sqrt{\sin^2(\alpha)+\gamma_T^2(1+ \cos(\alpha))^2}}
\end{equation}

and the impact parameter appears to be dependent on three parameters only:
\begin{equation}\label{d0_boosted}
  d_0'=r_T \sin(\alpha) \frac{\gamma_T}{\sqrt{\sin^2(\alpha)+\gamma_T(1+\cos(\alpha))^2}}= d_0 \frac{\gamma_T}{\sqrt{\sin^2(\alpha)+\gamma_T(1+\cos(\alpha))^2}}
\end{equation}

From eq.(\ref{d0_boosted}) and eq.(\ref{d0_init}) one can see that the value of $d_0$ depends on the initial kinematics ($\alpha,\ \sin(\theta^+)$), the boost factor of the charged mediator $\gamma_T$ (which is a function of $m_C$ only) and the lifetime $l_{\chi^+}$. The latter depends mostly on the coupling constant ($l_{\chi^+} \propto 1/\mu^2$) and the mass splitting $\Delta m_{lC}$ (which has one-to-one correspondence with the transverse momentum $p_T$), with a mild dependence on $m_C$.

\section{Numerics}

Using eq.(\ref{d0_init}), one can try and calculate expected $d_0$ for the particular parameter point, knowing the lifetime of the charged mediator:

\begin{equation} \label{d0_lifetime}
  d_0= \beta \gamma\ c \tau_{\chi^+} \sin(\theta^+) \sin (\alpha)
\end{equation}

where $\gamma$ is a full boost factor of $\chi^+$ and parameters $\sin(\theta)=p_T(\chi^+)/p(\chi^+), \alpha$ can be found from simulations.

Let us have a look at the typical of the relevant parameters for $m_C=324$: from fig.(\ref{fig:charged-kin}) one can deduce that $\sin(\theta) \simeq (100-250)/(250-500)=0.2-1$ and $\gamma=(1-\beta^2)^{-1/2}=(1-(p/E)^2)^{-1/2}=(1-(0.6-0.8)^2)^{-1/2}=1.2-1.7$ (so $\gamma \beta \sim 0.7-1.4$), from fig.(\ref{fig:alpha}) -- that $\sin \alpha \simeq \alpha \simeq 0-0.5$. This gives an estimation for the impact parameter $d_0 \simeq (0-0.7)c \tau_{\chi^+}$. For the coupling we used before ($\mu/v=9 \times 10^{-6}$), $c \tau = 0.1 \text{cm}$; for the new ones ($\mu/v=2 \times 10^{-6}$) - $c \tau = 2 \text{ cm}$. This means that for the largest lifetime possible in our scenario I expect that the the large fraction of event will be in the region of $d_0 \sim 1 \text{cm}$.

\begin{figure}[h!]
\centering
\includegraphics[width=0.45 \textwidth]{histos/ptcharged_different_mC}
\includegraphics[width=0.45 \textwidth]{histos/p_charged_different_mC}


\caption{\label{fig:charged-kin}Transverse and full momentum of the charged mediator, second row: rapidities of $m_C$.}
\end{figure}


\begin{figure}[h!]
\centering
\includegraphics[width=0.45 \textwidth]{histos/dphi-psiminus-lminus_different_mC.png}\\
\includegraphics[width=0.45 \textwidth]{histos/dphi-psiminus-lminus_mC324_different_splittings.png}
\includegraphics[width=0.45 \textwidth]{histos/dphi-psiminus-lminus_mC120_different_splittings.png}

\caption{\label{fig:alpha}Values of $\alpha$ for a fixed mass splitting and various $m_C$, as well as for two fixed values of $m_C$ and various splittings.}
\end{figure}

These calculations, however, do not take into account the Poisson statistics for the decaying particle. So, in reality, the probability that the particle with the lifetime $c \tau$ would travel the distance d, is

\begin{equation} \label{Poisson}
  P(d)=e^{-\frac{d}{\beta \gamma c \tau}}
\end{equation}

This results in the exponential suppression of number of events with large displacements.

Let us now have a look at the numerical results for the "currently established benchmark point":
{\textbf BP } $m_l = 304 \text{ GeV},\ m_C = 324\text{ GeV}, \mu/v=2 \times 10^{-6} (\text{or } rr=10^{-8} \text{ in the cards})$.
 The relevant boost factors combination $\beta \gamma$ and the travel distance of $\chi_+$ are shown on fig.(\ref{fig:boost-distance}). Note that the second plot has logarithmic scales on both axes, so it has exactly the exponential shape discussed in \ref{Poisson}.

\begin{figure}[h!]
\centering
\includegraphics[width=0.45 \textwidth]{histos/betagamma_rr10-8}
\includegraphics[width=0.45 \textwidth]{histos/distance_rr10-8}


\caption{\label{fig:boost-distance}.Left: boost factors combination $\beta \gamma$, right: travel distance of $\chi_+$.}
\end{figure}

The impact parameter, calculated using HEPMC output generated by Pythia8 is shown on fig.(\ref{fig:d0}). The first plot has no pT cuts, and the difference between electrons and muons occurs due to the presence of large number of soft electrons originating from the primary vertex. This difference is gone once the cut of $pT=10 \text{ GeV}$ is applied, which can be seen on the second plot. The shape of the d0-histograms suffers from exactly the same suppression as discussed above.

\begin{figure}[h!]
\centering
\includegraphics[width=0.45 \textwidth]{histos/d0_ctau2cm}
\includegraphics[width=0.45 \textwidth]{histos/d0_ctau2cm_pTcut}


\caption{\label{fig:d0}.d0 of electrons and muons. Left: no pT cuts, right: $pT=10 \text{ GeV}$ cut is applied}
\end{figure}

\section{Dependence on model parameters}

From eq.(\ref{d0_lifetime}) one deduce how does the impact parameter depend on the model parameters: $\gamma_T,\ \theta, \alpha$ depend only on $m_C$ (for the latter see fig.(\ref{fig:alpha})), and the lifetime is mostly dependent on the coupling and the mass splitting:

\begin{equation}
  c \tau \simeq \frac{v^4}{\mu^2 \Delta m_{Cl}^3}
\end{equation}

This happens simply due to the change of $c \tau$, so if we changed the coupling constant such that the lifetime is $2 cm$ istead of $0.1 cm$ before, we expect the same change in $d0$: fig.(\ref{fig:d0-different-couplings}}). Unfortunately, there is not enough statistics on the second plot, so the value we get is true up to an order of magnitude.

\begin{figure}[h!]
\centering
\includegraphics[width=0.45 \textwidth]{histos/d0_ctau2cm}
\includegraphics[width=0.45 \textwidth]{histos/d0l_ctau01}


\caption{\label{fig:d0-different-couplings}.d0 of electrons and muons for different lifetimes.}
\end{figure}

\end{document}
