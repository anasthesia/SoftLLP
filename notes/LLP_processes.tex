\documentclass[12pt,letterpaper,notitlepage]{article}

\usepackage{epsfig,amsmath,amssymb,cite,color,multirow,ulem,bm}

\usepackage{graphicx,wasysym}
\usepackage{hyperref}
\usepackage{tabulary}

\usepackage{pdfpages}
\usepackage{xcolor}

\newcolumntype{K}[1]{>{\centering\arraybackslash}p{#1}}

\makeatletter
\newcommand{\thickhline}{%
    \noalign {\ifnum 0=`}\fi \hrule height 1pt
    \futurelet \reserved@a \@xhline
}
\newcolumntype{"}{@{\hskip\tabcolsep\vrule width 1pt\hskip\tabcolsep}}

\newcommand{\listintertext}{\@ifstar\listintertext@\listintertext@@}
\newcommand{\listintertext@}[1]{% \listintertext*{#1}
  \hspace*{-\@totalleftmargin}#1}
\newcommand{\listintertext@@}[1]{% \listintertext{#1}
  \hspace{-\leftmargin}#1}
\makeatother

\oddsidemargin 0cm
\evensidemargin 0cm
\marginparwidth 68pt
\marginparsep 10pt
\topmargin 0cm
\headheight 0pt
\headsep 0pt
\footskip 30pt
\textheight 22cm
\textwidth 16.5cm
\columnsep 10pt
\columnseprule 0pt

\newcommand{\tev}{\,\, \mathrm{TeV}}
\newcommand{\gev}{\,\, \mathrm{GeV}}
\newcommand{\mev}{\,\, \mathrm{MeV}}



\begin{document}

\begin{center}
\LARGE Singlet-triplet fermion dark matter\\
\Large LLP processes
\end{center}

\vspace{1.0cm}
\begin{abstract}
\vspace{0.2cm}\noindent
This file contains a list of the processes which are expected to give interesting LLP signatures in our model \cite{Filimonova:2018qdc}, possible backgrounds and signal features.
\end{abstract}

\section{Two displaced leptons}

The production goes trough Z-boson, leading to a signature with two displaced leptons of the opposite sign (Fig. \ref{fig:2displaced_diagram}). The displacement is caused by the mixing angle suppression $\theta \approx \mu/(m_T - m_S)$, the mass splitting between the charged and the light neutral states is $\Delta m_{c \ell} = 15-30 \text{ GeV}$. The lifetime of the charged state varies in the range $\c \tau_+ \simeq 0-1.5(4) \text{ cm}$ in the scalar(pseudo-scalar) scenario and the portal coupling is $\mu/v \lesssim 3\times 10^{-5}\ (3\times 10^{-4})$ in the scalar (pseudo-scalar) model.

\subsection{Main backgrounds expected}

\begin{itemize}

\item 2015 CMS $e-\mu$-search for the opposite sign leptons with large impact parameters $d_0= 0.02 - 2 \text{ cm}$, $\c \tau_{\tilde{t}}=10^{-2}-10^2 \text{ cm}$ \cite{Khachatryan:2014mea}: HF states with $\tau \simeq 0.05 \text{ cm/c}$ and $Z \rightarrow \tau \tau$ decays with $τ \simeq 0.0087 \text{ cm/c}$ (Fig. \ref{fig:CMS2015-bkg}).

\begin{figure}[!h]
\centering
\includegraphics[height=3.0in]{CMS2015-bkg-e}
\includegraphics[height=3.0in]{CMS2015-bkg-mu}
\caption{\label{fig:CMS2015-bkg} Backgrounds from \cite{Khachatryan:2014mea}}
\end{figure}

\item 2016 CMS $e-\mu$-search for the opposite sign leptons with $d_0= 200 \text{ \mu m} -
10 \text{ cm}$, $\c \tau_{\tilde{t}}=10^{-2}-10^2 \text{ cm}$ \cite{CMS:2016isf}: decays of τ leptons and B or D mesons
with $c \tau_\tau \simeq 87 \text{ \mu m} ,\ c \tau_B \simeq 500 \text{ \mu m},\ c \tau_D \leq 100 \text{ \mu m}$ (Fig. \ref{fig:CMS2016-bkg}).

\begin{figure}[!h]
\centering
\includegraphics[height=3.0in]{CMS2016-bkg-e}
\includegraphics[height=3.0in]{CMS2016-bkg-mu}
\caption{\label{fig:CMS2016-bkg} Backgrounds from \cite{CMS:2016isf}}
\end{figure}

\item 208 CMS soft prompt oppositely charged (e or $\mu$) leptons \cite{Sirunyan:2018iwl}: ``The main backgrounds arise from events in which one of the leptons is not prompt (mainly
from W+jets events), events from fully leptonic tt decays ($t\bar{t}(2l)$), and Drell–Yan (DY) processes with subsequent decays $\gamma/Z^* \rightarrow \tau \tau \rightarrow ll \nu_l \nu_l \nu_\tau \nu_\tau$. Smaller backgrounds are from tW production (tW) and the diboson processes WW and $ZZ^*$, with $Z^* \rightarrow ll$ and $Z → \nu \nu$ (VV). Processes
such as $t\bar{t}W,\ t\bar{t}Z, WWW, ZZZ, WZZ$ and WWZ as well as processes including the Higgs boson have very small contributions, and are grouped together as “Rare”. ''

\end{itemize}

\begin{figure}[!h]
\centering
\includegraphics[height=3.0in]{displaced-leptons.png}
\caption{\label{fig:2displaced_diagram}New signature of two displaced soft leptons at the LHC.}
\end{figure}

\section{One displaced and one prompt lepton}

The production goes trough W-boson, leading to a signature with one displaced and one prompt lepton of the same or opposite signs (\ref{fig:2displaced_diagram}). The displacement on the long-branch side is caused by the mass splitting suppression $\Delta m_{hc} = m_h - m_c \approx \mu^2/(m_T - m_S) - 160\,\text{MeV}$ .~\footnote{Caution: this approximate formula is valid only in the small $\mu$ regime, however, $\mu$ is sizeable in the considered case} (we expect pion to be very soft, so that it is not observed), the second branch is prompt due to a relatively large mixing angle. The mass splitting between the charged and the light neutral states is again $\Delta m_{c \ell} = 15-30 \text{ GeV}$, and between the heavy neutral and the charge states: $\Delta m_{h c} \simeq 160 \text{ MeV}$. The lifetime of the charged state varies in the range $\c \tau_+ \simeq 0-0.5 \text{ cm}$ and the portal couplings $\mu/v \sim 10^{-2}\ (3\times 10^{-2}-10^{-1})$ in the scalar (pseudo-scalar) model.

\subsection{Expexted backgrounds}
Only non-symmetric processes:
\begin{itemize}
  \item W+jets events (?)
  \item tW production (?)
\end{itemize}

\begin{figure}[!h]
\centering
\includegraphics[height=3.0in]{displaced+prompt-leptons.png}
\caption{\label{fig:2displaced_diagram}New signature of one displaced and one prompt soft leptons at the LHC.}
\end{figure}

\begin{thebibliography}{6}

\bibitem{Filimonova:2018qdc}
  A.~Filimonova and S.~Westhoff,
  %``Long live the Higgs portal!,''
  arXiv:1812.04628 [hep-ph].
  %%CITATION = ARXIV:1812.04628;%%

  \bibitem{Khachatryan:2014mea}
    V.~Khachatryan {\it et al.} [CMS Collaboration],
    %``Search for Displaced Supersymmetry in events with an electron and a muon with large impact parameters,''
    Phys.\ Rev.\ Lett.\  {\bf 114}, no. 6, 061801 (2015)
    doi:10.1103/PhysRevLett.114.061801
    [arXiv:1409.4789 [hep-ex]].


  \bibitem{CMS:2016isf}
    CMS Collaboration [CMS Collaboration],
    %``Search for displaced leptons in the e-mu channel,''
    CMS-PAS-EXO-16-022.

\bibitem{Sirunyan:2018iwl}
  A.~M.~Sirunyan {\it et al.} [CMS Collaboration],
  %``Search for new physics in events with two soft oppositely charged leptons and missing transverse momentum in proton-proton collisions at $\sqrt{s}=$ 13 TeV,''
  Phys.\ Lett.\ B {\bf 782}, 440 (2018)
  doi:10.1016/j.physletb.2018.05.062
  [arXiv:1801.01846 [hep-ex]].

\end{thebibliography}

\end{document}
